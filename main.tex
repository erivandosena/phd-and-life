%%%%%%%%%%%%%%%%%%%%%%%%%%%%%%%%%%%%%%%%%%%%%%%%%%%%%%%%%
%                                                       %
%                  Copyright (C) 2024 by:               %
%                                                       %
%                     Erivando Sena                     %
%                <erivandosena@gmail.com>               %
%                                                       %
%%%%%%%%%%%%%%%%%%%%%%%%%%%%%%%%%%%%%%%%%%%%%%%%%%%%%%%%%
%                                                       %
% This work may be distributed and/or modified under    %
% the conditions of the LaTeX Project Public License,   %
% either version 1.3 of this license or (at your        %
% option) any later version.                            %
% The latest version of this license is in              %
% http://www.latex-project.org/lppl.txt                 %
% and version 1.3 or later is part of all               %
% distributions of LaTeX version 2005/12/1 or later.    %
%                                                       %
%%%%%%%%%%%%%%%%%%%%%%%%%%%%%%%%%%%%%%%%%%%%%%%%%%%%%%%%%

\documentclass[twocolumn]{article}
\usepackage[a4paper, margin=1in]{geometry}
\usepackage{graphicx}
\usepackage{titlesec}
\usepackage{titling}
\usepackage{abstract}
\usepackage{adjustbox}
\usepackage{everypage}
\usepackage{fancyhdr}
\usepackage{lastpage}
\usepackage{hyperref}
\usepackage{tikz} % For absolute positioning

% Define the title and author of the paper
\title{How To Get a Ph.D. and Have a Life, Too}
\author{Richard E. Baker}
\date{July 1997 \\\textit{Volume 29, Number 3, SIGCHI Bulletin}}

% Define the abstract text
\renewcommand{\abstractname}{}
\renewcommand{\absnamepos}{empty}

% Define the column separation and line width
\setlength{\columnsep}{0.5in}
\setlength{\columnseprule}{0.2pt}

% Define title and section formatting
\titleformat{\section}{\bfseries\large}{\thesection}{1em}{}
\titleformat{\subsection}{\bfseries}{\thesubsection}{1em}{}
\pretitle{\begin{center}\LARGE\bfseries}
\posttitle{\par\end{center}\vskip 0.5em}
\preauthor{\begin{center}\large}
\postauthor{\par\end{center}}
\predate{}
\postdate{}

% Define the header and footer
\pagestyle{fancy}
\fancyhf{} % Clean cabeçalho and rodapé
\fancyhead{}
\fancyfoot{}
\fancyhead[C]{\textit{Volume 29, Number 3, SIGCHI Bulletin}}
\fancyfoot[C]{\thepage}
\renewcommand{\headrulewidth}{0pt}
\renewcommand{\footrulewidth}{0pt}

% Setup rodapé na última página
\fancypagestyle{lastpage}{
  \fancyfoot[C]{%
      \raisebox{-1ex}{\hspace{0cm} \thepage}%
      \\
      \raisebox{-1ex}{ \href{https://github.com/erivandosena/phd-and-life-latex} {\LaTeX\ Version By: Erivando Sena}}%
  }
}

% Apply estilo na última página
\pagestyle{fancy}
\AtEndDocument{\thispagestyle{lastpage}}

\begin{document}

\maketitle

% \begin{abstract}
% \noindent Texto do abstract aqui.
% \end{abstract}


% Add the "Students" no cabeçalho da primeira página
\AddEverypageHook{%
  \ifnum\value{page}=1%
    \begin{tikzpicture}[remember picture,overlay]
      \node[anchor=north west,rotate=90,font=\fontsize{55}{60}\selectfont\bfseries,text depth=0pt]
      at ([xshift=0.8cm,yshift=-15.8cm]current page.north west) {Students};
    \end{tikzpicture}
  \fi
}

% The rest of your document goes here
% \section*{Introductin}
\noindent
Did you ever wonder how you can get more done in less time? How can you get your
assignments completed, meet with your study group, spend time on research,
complete the exercises, write the required papers before the deadline, and
continue to develop your dissertation? Where will you find time for your job with
all the necessary courses, research, and documentation? If you have a family, how
will you fit time for your spouse, children, and home life into your already busy
schedule? Where do you find the time for shopping, cleaning, fixing, running, and
most importantly, time with the children? Lastly, how will you possibly be able
to relax and get some rest?

The answer is effective time management.

\section*{Time Masters}
\noindent
After many years of observing individuals in industry and academia who have
successfully mastered the art of time management, the author wished to share with
other students some of the more effective techniques. The one covered here is a
combination of the most commonly used techniques. The central theme of success for
time masters is the ability to juggle the many demands on their time with the
limited time in their schedules.

Everything we do is a function of time management; whether we schedule it or
another does. Everything that happens is a function of timing, whether ours,
someone else's, or just a function of nature. Time is the logical river in which
we live and it flows continually, unstopped by any dam. It has become a personal
and psychological medium within which each of us experiences our own internal
time and an external time imposed by society, schedules, deadlines, work, and
academia. (Ventura, 1995). One method by which we can be freed from these demands
is to withdraw from society to a log cabin in the woods – or we can become adept
at time management.

\section*{Why Time Management is Important}
\noindent
One of the most prevalent, but underestimated, problems for graduate students is
time management (Quilty, 1996). Demands on one's time seem to increase while the
available time to accomplish things seems to decrease. The idea, of course, is to
never let the demands for time become equal to or greater than the time available.
Occasionally, however, this does happen and your prowess as a time master will be
tested. Your ability to manage your time will greatly influence your future success
in academia, industry, and society.

\section*{Factors of Time}
\noindent
The pressures to manage time have not always been a factor in society. Before the
railroads began to run over one hundred years ago, time was not as important as
it is to people today. Before this, local time in New York did not relate to any
particular time in Chicago, Los Angeles, Tokyo, or even in London. It was not
important to have everyone on a coordinated time. (Ventura, 1995). As our society
has become more industrialized, our focus on time has become more intensified.
Today, in our technically advanced and networked world community, time is
essential to the coordination of meetings and events of all kinds.
Almost everything we do is set to a time schedule. The result: people who feel
that they are being driven by the schedule instead of setting it. (Nickerson, 1992).

\section*{Getting Control}
\noindent
How can you identify yourself as an ineffective time manager? If you experience
any of the following, you may have time management problems: constant rushing,
frequent lateness, low productivity, low energy and motivation, frustration with
schedules or deadlines, impatience, chronic vacillation between alternatives,
difficulty setting and achieving goals, or procrastination (Covey, 1989).
Everyone is allocated the same number of hours in a day. Yet how do some people
feel like 24 hours is not enough while others manage to get their work done and
still have time left over to enjoy themselves? People who effectively manage
their time have learned to put a little discipline and structure into their lives.
They focus most of their time and energy on what is most important to them.
They minimize the time they spend on activities they see as personally
unimportant. Essentially, by putting a little structure into their lives,
they have learned to manage themselves (Covey, 1989) and focus their available
time on the activities which they see as important. Their greatest tools are
planning, setting priorities for activities, juggling the demands into an
overbooked schedule, defeating procrastination, and keeping focus on what is
important to them.

\section*{The Key is Planning}
\noindent
The key to effective time management is planning. This involves a series of goal
and priority setting. Time planning is an exercise in matching the activities
you either must accomplish or wish to accomplish with the time you have available.
The first step is in setting the overall goals and determining the steps or
activities you need to accomplish to reach the goal. Once the activities are
identified, priorities can be assigned to each. Then, using the priorities,
match the sequence of activities to the allotted time in your schedule.
This allows a managed approach.

Effective time masters use a combination of time plans; short term and long term,
daily, semester, and academic career. The short term plans should supplement or
assist the achievement of the long term plans. Then, by focusing their energies
on the priorities of the tasks in each plan, the larger goals can be met.

Take a few minutes to write down your goals for your academic student career.
This can probably be stated in one or two sentences. Then, write down your goals
for each academic period to help achieve your career goal. Then, write down your
necessary goals or activities for the current academic period in chronological
order, everything you must accomplish as exercises, assignments, papers, tests.
Next, write down the activities or goals for the current week. Some of the items
on the current period list may be included if they fall into this week.

Lastly, write down your goals or activities for the day. This is the first step
in learning to juggle your time demands. Make sure to include all the tasks you
need to perform today. Remember to include the family, work, and social events.
Now, assign times to these tasks.

Obviously, there will be days in which the activities fill more time than is
available on your timeline. So, one must set priorities: what must be completed;
what is good to have completed; and what can wait until another day.
As stated above, concentrate on the important activities and accomplish tasks
according to priority.

\section*{Setting Priorities}
\noindent
Perhaps all your activities are important. No one ever has the ability to do only
the ones they please. But, which to do first? To help define priorities, review
each task and decide its manner of importance to you and the urgency with which
it must be accomplished (Covey, 1994). Then assign the task to one of the
following priorities: 1, 2 or 3.

A Priority 1 is a task you must accomplish today. It is important to you and it
is urgent (Covey, 1989). Sometimes, other people will impose these tasks upon you.
Sometimes they are simply a carry-over from a previous day, but the task’s
deadline is near. For example, a forgotten assignment that is due; or, your
spouse cannot pickup the children after work; or, one of the children is ill and
needs to go to the doctor. Priority 1 tasks must be accommodated into your day
and often make you feel as if you are in crisis management. Obviously, you should
work to eliminate Priority 1 tasks from your list whenever possible.

Priority 2 tasks are items that are of greater importance than urgency
(Covey, 1989). Priority 2 tasks advance you toward your greater goals. They are
characterized by your planning, balance in your schedule, and discipline toward
accomplishment. For example, all tasks toward your overall goal of completing
your dissertation should fall into category 2; plan time for doing your email
and phone messages; spending time with that special someone in your life or with
your family, and, vacation is a planned and important activity. The effective
time manager’s goal is to always work with Priority 2 tasks.

Last is priority 3 tasks. These tasks are usually not important to you; but are
urgent for someone else. This is the priority where tasks go that are due to a
lack of planning by someone else. Their lack of planning does not justify an
emergency for you. Priority 3 tasks are typified by ringing phones with
solicitors, inane email, and unproductive meetings. Work priority 3 tasks
into your daily list when you have extra time (maybe never).

\section*{Plan your Daily Schedule}
\noindent
One of the best tools is the individual's daily to do list. Write down a list of
the activities you must accomplish today. This is usually all the items in
Priority 1. Then, review your list of Priority 2 and 3 tasks. Add any of these
that must be accomplished today, or for which you have planned time, to the list.
Next, assign times to the activities and place them in chronological order for
the day. Assign times by priority and begin compromising with schedule, location,
importance, and the many other factors with each activity. This planning
activity is the juggling. The final list becomes the planned timeline for the day.

Review the list for any changes in priority. Priority 1 tasks should not be
carried over past the completion date; they should be finished. Priority 2 items
should be on the list and completed by their planned day for accomplishment.
Consider deleting Priority 3 items from your list when you are sure they will not
become Priority 1.

\section*{Use a Mind Map}
\noindent
The daily list is usually accomplished at the same time of day each day.
Most time masters do it at the end of the day so they have the evening to review
the activities in their minds and mentally prepare for the next day. Some do it
early in the morning so they have the timeline fresh in their minds as they
begin the day. This is building and using a mind map: a logical view of the
activities, their sequence, and locations for accomplishment of the next day. It
helps time managers prepare themselves, prepare their resources, and meet the
challenges of the busy day just a little more effectively.

The next step is to work your planned schedule. However, the time master is
always prepared for interruptions or changes (Ventura, 1995).

\section*{Review the Goal Lists}
\noindent
Keep the long and short term goal lists you created earlier. The first list,
your academic student career, should change only slightly before you graduate.
Major changes will be in research area, not necessarily in major discipline. The
second list, the academic period list, will change only to reflect your changes
in study or research. The third list of the current academic period will change
to reflect any new and important activities you add to your family, social, work,
or academic calendars. Make sure to plan them appropriately as Priority 2 tasks.
The weekly list will change more rapidly and will often be dropped after the
time master becomes comfortable with using the daily lists in conjunction with
the academic period list. The daily list becomes the action plan to execute the
larger, multi-week plan of the academic period. Thus, the time master really
works from two lists, the daily and academic period, and keeps the others as a
long range planning maps.

\section*{Defeat Procrastination}
\noindent
A few simple words about the time manager's greatest enemy: procrastination.
These simple words are: do it now! Use a few tricks to help yourself stay away
from the desire to put it off. Schedule nasty jobs for specific times and reward
yourself upon completion with some special treat. Think about how good it will
feel to have the job completed instead of dreading to do it; and then having to
work overtime to complete it. Sometimes, students become overwhelmed by the size
of the overall job. If the job is too big, break it into manageable pieces.
Once it is in manageable sizes, the individual parts can be addressed and worked
into an overall plan of action.

\section*{Help Yourself}
\noindent
Ask yourself this question: "Is what I am doing at this time moving me towards
the accomplishment of my tasks?" If not, stop wasting time on it. Time is a
renewable resource and is flowing constantly. Why waste this valuable resource
on things that do not aid you? Keep your focus on activities that truly help you
accomplish your short-term and long-term goals.

An often overlooked item is one’s prime time of day. Select your prime time of
best operation, whether it is morning, afternoon, or evening. Use this time to
your best advantage because it will be your most productive. A time manager
often utilizes this portion of the day to accomplish most of the tasks on their
daily list. It will be to your benefit to know when you are most effective.

When possible, set aside dedicated time periods each day to accomplish routine
activities such as e-mail, study, and time with the family. This allows one to
focus while keeping routine matters from expanding to fill the entire day.

One of the largest contributors to time management failure is overly aggressive
planning. Setting overwhelming lists and accomplishing only a small percentage
of tasks will make you feel depressed and unproductive. Instead, set conservative
plans so that accomplishing some major percentage of the tasks will allow you
to feel productive and eager to meet the challenge again the next day. A sense
of accomplishment helps you to feel in control of your time.

\section*{Flexibility – the Most Important Attribute}
\noindent
As everyone knows, no plan is executed without some change. So will be your
action plans. Effective time masters always allow for change. They build extra
time into their action plans for unplanned interruptions. As a fact of life,
everyone has them so you need to plan accordingly. There are always those little
interruptions such as: your dissertation advisor; the dean called; your family
has an emergency; or any of a multitude of other events of daily life.

The critical success factor to time management is flexibility. You must be able
to set goals and priorities; and, then build an execution plan with schedule.
Secondly, you need to work that plan to the best of your abilities within the
time you have allotted. Lastly, plan for change. Anticipate that something or
things will interrupt or cause change in your daily list of tasks. Allow at
least one half hour for unexpected events and make sure it is part of every
day’s plan.

A key element of managing the daily plan is the ability to reassess and modify
your objectives. By periodically reevaluating the plan, deviations can be
minimized and the successes of the day can be assured. Be flexible and modify
your plan to still meet your overall objectives. Every day brings you one step
closer to your goal.

\section*{Conclusion}
\noindent
The author offers this article with the hope that other students will benefit
from the experiences of others who have already joined the ranks of effective
time managers. The ability to manage one's time and many demands will determine
the success of the individual in the future, regardless of path. The time master
is the individual who follows a daily plan; never forgets family or work as a
major part of that plan; and always allows for interruptions by family, friends,
and peers at work. A critical source of support, often forgotten by busy
students during their research, is their family and their friends.

Lastly, remember to: Plan your work. Work your plan. Plan for change.

\section*{About the Author}
\noindent
Richard E. Baker is the Manager of Human Factors in Computing at Electronic Data
Systems Corporation. He is also a doctoral student in the Department of Computer
and Information Sciences at Nova Southeastern University in Fort Lauderdale,
FL, USA. His research interest is in the semiotics of information systems.
Richard can be reached at bakerr@scis.nova.edu or dbaker@mig.eds.com.

% References Section
\begin{thebibliography}{9}
  % Add additional references here
  \bibitem{covey1989} Covey, S. R. (1989). \textit{The Seven Habits of Highly
  Effective People: Restoring the Character Ethic}. New York, NY: Simon \& Schuster Inc.
  \bibitem{covey1994} Covey, S.R. (1994). \textit{First Things First: To Live, to
  Love, to Learn, to Leave a Legacy}. New York, NY: Simon \& Schuster Inc.

  \bibitem{nickerson1992} Nickerson, R.S. (1992). \textit{Looking Ahead: Human
  Factors Challenges in a Changing World}. Hillsdale, NJ: Lawrence Erlbaum Associates, Publishers.

  \bibitem{quilty1996} Quilty, S.M. (1996, November). Managing time and workload
  in college: A weight and balance problem. \textit{Flight Training}, 48-49.

  \bibitem{ventura1995} Ventura, M. (1995, January-February). Prisoners of time:
  The age of interruption. \textit{Networker}, 19-31.
\end{thebibliography}

\end{document}